\section{Numerical code}
%---------------------------------------------------------------------------------------------------

\subsection{Overview}

This experiment was performed with version <NEMO_VER> of NEMO, based on revision <DCM_REV> of the Drakkar Configuration Manager (DCM). CPP keys used for compilation are:
\input{./TexFiles/cpp}

%---------------------------------------------------------------------------------------------------

\subsection{Ocean details}

\subsubsection{Vertical physics}

\paragraph{TKE scheme \\}

TKE is used to determine the vertical diffusion coefficient. The relevant namelist data are indicated below. 
Note that in this version, a non-standard treatment is performed on ice-covered area: (a) The background avt 
coefficient is divided by 10 under ice. (b) There is no background of Tke under ice. (c) The coefficient for 
surface input of tke (ebb) is reduced from 60 (open ocean) to 3.75 (ice covered regions). (d) Lang-Muir cells 
parametrization is turned off below ice. \\

For evd mixing coefficient we have \textit{rn\_avevd=10$m^2/s$} and we don't have any good reason to change it. \\

We keep \textit{nn\_mxl=3} for mixing length (Gurvan's advice).
For the TKE penetration below mixed layer we keep \textit{nn\_htau=1} with 0.5m at the equator to 30m 
poleward of 40 degrees as we don't have any good reason to change it (at NOCS they decide to replace 
30m by 10m in the code; Gurvan suggests to use a monthly climatology of waves to switch between 10m 
and 30m according to the waves but we can't test this on an ORCA12 reference configuration). \\

%-----------------------------------namzdf--------------------------------------------
\namdisplay{namzdf}
\namdisplay{namzdf_tke}
%-------------------------------------------------------------------------------------

\subsubsection{Horizontal physics}

\paragraph{Tracers \\}

We use TVD advection scheme for tracer. \\

We use a laplacian isopycnal diffusivity for tracers. The diffusivity is proportionnal to the local 
grid size (it decreases poleward). The horizontal eddy diffusivity for tracers is reduced 
to 125 $m^2/s$ for ORCA12 configuration (compared to ORCA025 : 300 $m^2/s$). \\

%-----------------------------------namtra_adv--------------------------------------------
\namdisplay{namtra_adv}
\namdisplay{namtra_ldf}
%-------------------------------------------------------------------------------------

\paragraph{Momentum \\}

We use a vector form momentum advection scheme with energy and enstrophy (EEN) conserving conditions. \\

We use a bi-harmonic viscosity for the lateral dissipation. Note that in the ORCA12 configuration, 
the viscosity is reduced by a factor 14 compared to ORCA025 configuration. 
The viscosity is proportionnal to the grid size power 3.

%-----------------------------------namdyn_adv--------------------------------------------
\namdisplay{namdyn_adv}
\namdisplay{namdyn_vor}
\namdisplay{namdyn_ldf}
%-------------------------------------------------------------------------------------

\subsubsection{Bottom Boundary Layer}

We used bottom boundary layer parametrization (\cite{Beckmann1997}). 
Both diffusive and advective BBL parametrization is used for tracers. 
There is no advective BBL for momentum \footnote{We only apply the 
improvement on BBL parametrization concerning tracers, not momentum, proposed in \cite{Hervieux}.}.\\

\textcolor{blue}{D\'ecision (Jean-Marc): En fait pas de dynbbl} \\

\textcolor{blue}{JMM added ln\_kriteria for NEMO3.4 (but we don't know if it is working, need to check sign issue in F95), I have to do a commit before starting!}
%-----------------------------------nambbl--------------------------------------------
\namdisplay{nambbl}
%-------------------------------------------------------------------------------------

\subsubsection{Lateral boundary condition}

The lateral momentum boundary condition was set to 0 (free-slip condition) except locally where we use no-slip condition (reading a 2D file ln\_shlat2d=.true.). 
Romain Bourdall\'e-Badie created this 2D shlat file and Markus Scheinert validated it.
Figure \ref{shlat_v3} shows patches of no-slip condition: Mediterranean Sea, Indonesian ThroughFlow, 
Baffin Strait; and a patch with regular transition from free-slip to no-slip: South West Greenland.

\textcolor{blue}{Question: should we add Antarctic coast and Bering Strait in the no-slip patches ? If we do that we have a diplomatic problem because 
it is contrary to what we decided at DRAKKAR meeting: Markus was designed responsible for shlat2d validation...} \\

\textcolor{blue}{Decision (Bernard): On va prendre ce shlat v3 mais on va y ajouter d'autres patch no-slip a Bering et tout autour de l'Antarctique, aussi on va etendre le patch no-slip de SW Groenland} \\

\textcolor{blue}{Objection (Thierry): no-slip Antarctique why not mais il faudrait un run test d\'edi\'e ORCA025, on va pas mettre ca direct dans le ORCA12 62 ans de reference...} \\

\textcolor{blue}{Markus utilise bien v3 pour son run K003, si on ajoute les patchs Antarctique+Bering et qu'on modifie le patch SW Greenland, 
il faudra informer Markus et Romain qu'on d\'ecide de ne pas se coordonner avec eux.}

%-----------------------------------namlbc--------------------------------------------
\namdisplay{namlbc}
%-------------------------------------------------------------------------------------

\begin{figure}[H]
\begin{center}
\includegraphics[width=15cm]{FIGURES/shlat2d_ORCA12grid_19102011_v3.eps}
\caption{Lateral boundary condition as defined in shlat2d\_ORCA12grid\_19102011\_v3.nc. Contours are bathymetry.}
\label{shlat_v3}
\end{center}
\end{figure}

\textcolor{blue}{Figures \ref{shlat_v0}, \ref{shlat_v1}, \ref{shlat_v2} show the other 2D shlat that we gave up.} 

\begin{figure}[H]
\begin{center}
\includegraphics[width=15cm]{FIGURES/shlat2d_ORCA12grid_19102011_v0.eps}
\caption{Lateral boundary condition as defined in shlat2d\_ORCA12grid\_19102011\_v0.nc. Contours are bathymetry.}
\label{shlat_v0}
\end{center}
\end{figure}

\begin{figure}[H]
\begin{center}
\includegraphics[width=15cm]{FIGURES/shlat2d_ORCA12grid_19102011_v1.eps}
\caption{Lateral boundary condition as defined in shlat2d\_ORCA12grid\_19102011\_v1.nc. Contours are bathymetry.}
\label{shlat_v1}
\end{center}
\end{figure}

\begin{figure}[H]
\begin{center}
\includegraphics[width=15cm]{FIGURES/shlat2d_ORCA12grid_19102011_v2.eps}
\caption{Lateral boundary condition as defined in shlat2d\_ORCA12grid\_19102011\_v2.nc. Contours are bathymetry.}
\label{shlat_v2}
\end{center}
\end{figure}

\subsubsection{Bottom boundary condition}

Bottom boundary condition is exactly the same as ORCA025.L75-GRD100. 
We apply a nonlinear bottom friction rn\_bfri2=1.e-3 $m^2/s^2$. An enhanced bottom friction was applied 
over the Bering Strait (fig. \ref{bfrcoef2}) and over the Torres Strait (fig. \ref{bfrcoef1}). The bottom friction coefficient was read in the file 
orca12\_bfr\_coef\_MAL101.nc\footnote{This file was deduced from orca025\_bfr\_coef\_GRD100.nc, extending the ORCA025 domain over the ORCA12 domain 
(i.e. copying one grid point into nine grid points with chgrid.f90 tool).}
The enhanced bottom friction corresponds to the coefficient rn\_bfri2 multiplied by (1 + 2D bottom friction mask file multiplied by rn\_bfrien=10). \\

%-----------------------------------nambfr--------------------------------------------
\namdisplay{nambfr}
%-------------------------------------------------------------------------------------

\begin{figure}[H]
\begin{center}
\includegraphics[width=15cm]{FIGURES/bfr_coef2.eps}
\caption{Bottom friction coefficient in the vicinity of Bering Strait. Colors are bathymetry.}
\label{bfrcoef2}
\end{center}
\end{figure}

\begin{figure}[H]
\begin{center}
\includegraphics[width=15cm]{FIGURES/bfr_coef1.eps}
\caption{Bottom friction coefficient in the vicinity of Torres Strait. Colors are bathymetry.}
\label{bfrcoef1}
\end{center}
\end{figure}

\textcolor{blue}{Remarque: Markus n'a pas de mask pour la bottom friction dans son run K003.} \\

\subsubsection{Tracer damping strategy}

We use 3D restoring of temperature and salinity toward annual climatology of Gouretski and Koltermann (2004) in very specific areas. 
We use the same regional 3D damping as in ORCA025.L75-GRD100 simulation.
%-----------------------------------namtra_dmp--------------------------------------------
\namdisplay{namtra_dmp}
\namdisplay{namtsd}
%-------------------------------------------------------------------------------------

\textcolor{blue}{Question: Does nn\_hdmp=-2 option already implemented in NEMO3.4 or should we do it ?}

\paragraph{Regional 3D damping (semi-enclosed seas) \\}

Red + Black Sea, Persian Gulf: 3D T S restoring with a time scale of 180 days.

\paragraph{Downstream the overflows \\}

Gulf of Cadix downstream Gibraltar Strait in the depth range 600-1300m: time scale of 6 days. \\
Gulf of Aden downstream Bab-El-Mandeb: time scale of 6 days over the whole water column. \\
Arabian Gulf downstream of the Ormuz Strait : \textcolor{blue}{which time scale ?}

\paragraph{Antarctic Bottom Water restoring \\}

3D TS restoring (time scale of 2 years) in an area limited by the sigma-2=34.7 isopycnal, a depth greater than 1000m and south of 30S. \\

We are using ORCA12.L46\_dmp\_mask.nc as dmpmask.nc: i.e. \textit{includefile.ksh} file indicates:
\begin{verbatim}
AABW_DMP=1
WDMP=ORCA12.L46_dmp_mask.nc
\end{verbatim}

This ORCA12.L46\_dmp\_mask.nc file was created with \textit{cdfmaskdmp} cdftool using T/S Gouretski climatology 2004. \\

\textcolor{blue}{Question: Comment on fait pour definir plusieurs rn\_timsk en rn\_dep correspondant � chacune de ces zones ?}

%---------------------------------------------------------------------------------------------------

\subsection{Autres trucs}

\textcolor{blue}{J'ai not\'e aussi ITF tidal mixing parametrization. On en fait quoi ? Pour l'instant on attend toujours une r\'eponse de Florent Liard pour avoir M2 K1 sur une grille ORCA12...}

%---------------------------------------------------------------------------------------------------

\subsection{Ice details}

\subsubsection{EVP rheology}

The model used the LIM2 model, but with the Elasto-Visco-Plastic rheology. The ice-ocean coupling is done every 6 model steps (i.e. every 36min). \\

\textcolor{blue}{Question: which frequency for wind stress computing ?}

\subsubsection{Thermodynamics}

The standard LIM2 thermodynamics is used. The only change with respect to the standard code is the use of cloud cover files (synoptic) instead 
of a standard constant value for nebulosity. It has an impact on the net shortwave radiation which is not absorbed in the thin surface 
layer and penetrates inside the ice cover. \\

\textcolor{blue}{Note that the ice thickness for lateral accretion (\textit{hiccrit}) is 0.6 in the Northern and Southern Hemisphere.}

%-----------------------------------namicedyn--------------------------------------------
\namdisplay{namicedyn}
\namdisplay{namicethd}
%-------------------------------------------------------------------------------------


\textcolor{blue}{Remark: In namicedyn bloc Romain has difference with us (dm=0.0e+03; nbitdr=550; resl=1.0e-06; cw=1.0e-02; 
pstar=2.5e+04; creepl=5.0e-09; ahi0=20.e0; telast=240). Should we take the same namicedyn bloc as him ?} \\

\textcolor{blue}{Remark: Actuellement on a exactement la meme namelist\_ice que Markus pour K003} \\



