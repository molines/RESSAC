\section{Numerical code}

%---------------------------------------------------------------------------------------------------

\subsection{Overview}

This experiment was performed with version 3.4 of NEMO, with revision 1196 of the DRAKKAR Config Manager (DCM). CPP keys used for compilation are:
\input{./TexFiles/cpp}

%---------------------------------------------------------------------------------------------------

\subsection{Ocean details}

\subsubsection{Vertical physics}

\paragraph{Vertical mixing :TKE scheme, EVD and tidal mixing \\}

TKE is used to determine the vertical diffusion coefficient. The relevant namelist data are indicated below. 
Note that within DRAKKAR we use a non-standard treatment  on ice-covered area: (a) The background avt 
coefficient is divided by 10 under ice.  (b) The coefficient for 
surface input of tke (ebb) is reduced from 67.83 (open ocean) to 3.75 (ice covered regions). (c) Lang-Muir cells 
parametrization is turned off below ice. \\

In the mixing length calculation (a key variable in TKE scheme) \textit{nn\_mxl=3} is used, (the most sophisticated way for
bounding the mixing length scale). \\
 
The depth of penetration of surface tke is computed using \textit{nn\_htau=1} so that it increases from 0.5m at the equator to 30m 
poleward of 40 degrees (standard values). Some experiments were performed in the UK (NOCS and UKMO) reducing to 10m the maximum
value. A test of this modification performed with ORCA025 almost indicated a degradation (according to our standard) 
of the mixed layer depth in summer. Gurvan suggest that the good way to proceed is to use a map of this 
penetration depth, deduced from a wave monthly climatology. This is still to be done. \\

The vertical density instabilities are treated with enhanced vertical diffusivity (\textit{nn\_aevd=10$m^2/s$}) at the interface 
of two layers with density inversion. In this simulation we don't apply mixing on momentum (\textit{nn\_evdm=0}). 
For evd mixing coefficient we use the classical value \textit{rn\_avevd=10$m^2/s$}.\\

Vertical mixing produced by the tidal internal wave breaking is parameterized in the simulation (\textit{key\_zdftmx} defined). 
(Bessi\`eres et al., 2008\cite{Bessiere_al_GRL08}, Koch-Larrouy et al., 2007\cite{Koch-Larrouy_al_GRL07}).
In this simulation, the parameterization formely derived for the Indonesian
Through Flow region was also applied to the Solomon Sea (as suggested by \cite{Melet} ) and to the strait of Gibraltar. 

%-----------------------------------namzdf--------------------------------------------
\namdisplay{namzdf}
\namdisplay{namzdf_tke}
\namdisplay{namzdf_tmx}
%-------------------------------------------------------------------------------------
\textcolor{blue}{ JM : met on une carte du masque ITF etendu ? }

\subsubsection{Horizontal physics}

\paragraph{Tracers \\}

As in all the DRAKKAR simulation so far, TVD advection scheme for tracer is used. \\

Tracer diffusion is performed with an isopycnal laplacian operator. The standard isopycnal slope computation is used.
The diffusivity coefficient is proportionnal to the local grid size (it decreases poleward) and the maximum value at the
equator is 100 $m^2/s$, in total agreement with the value used in ORCA025 ( 300 $m^2/s$). \\

%-----------------------------------namtra_adv--------------------------------------------
\namdisplay{namtra_adv}
\namdisplay{namtra_ldf}
%-------------------------------------------------------------------------------------

\paragraph{Momentum \\}

Vector form momentum advection scheme with energy and enstrophy conserving conditions (EEN) is used. This simulation
was realized before Nicolas Ducousso findings in the COMODO project. He found (1) that the lateral boundary condition along
the model coast line has to be corrected  and (2) that the used numerical stencil in EEN may trigger an Hollinsworth instability
impacting the way eddies are generated and propagates. This will be corrected in next round of simulations. \\

Lateral momentum dissipation is achieved using an horizontal bi-harmonic operator. The hyper viscosity coefficient is proportional
to the cube of the model grid size. The maximum value (absolute value) at the equator is  $1.25^{10} m^4/s$.

%-----------------------------------namdyn_adv--------------------------------------------
\namdisplay{namdyn_adv}
\namdisplay{namdyn_vor}
\namdisplay{namdyn_ldf}
%-------------------------------------------------------------------------------------

\subsubsection{Lateral boundary condition}

The lateral momentum boundary condition was set to 0 (free-slip condition) except locally where we use no-slip condition (reading a 2D file ln\_shlat2d=.true.). 
Romain Bourdall\'e-Badie created this 2D shlat file and Markus Scheinert validated it.
Figure \ref{shlat_v3} shows patches of no-slip condition: Mediterranean Sea, Indonesian ThroughFlow, 
Baffin Strait; and a patch with regular transition from free-slip to no-slip: South West Greenland.

%-----------------------------------namlbc--------------------------------------------
\namdisplay{namlbc}
%-------------------------------------------------------------------------------------

\begin{figure}[H]
\begin{center}
\includegraphics[width=15cm]{./TexFiles/Figures/<CONFIG_CASE>_shlat2d}
\caption{Lateral boundary condition as defined in shlat2d\_ORCA12grid\_19102011\_v3.nc. Contours are bathymetry.}
\label{shlat_v3}
\end{center}
\end{figure}

\subsubsection{Bottom Boundary Layer}

We used bottom boundary layer parametrization of Beckmann and D\"oscher (2004)\cite{Beckmann1997}. 
Both diffusive and advective BBL parametrization is used for tracers. 
There is no advective BBL for momentum \footnote{We only apply the 
improvement on BBL parametrization concerning tracers, not momentum, proposed in Hervieux (2007)\cite{Hervieux}.}.\\

%-----------------------------------nambbl--------------------------------------------
\namdisplay{nambbl}
%-------------------------------------------------------------------------------------


\subsubsection{Bottom Friction}

A classical quadratic bottom friction is used, with a drag coefficient of rn\_bfri2=1.e-3 $m^2/s^2$. 
An enhanced bottom friction was applied over the Bering Strait (fig. \ref{bfrcoef2}) and over the Torres Strait (fig. \ref{bfrcoef1}) (mask read in 
the file orca12\_bfr\_coef\_MAL101.nc ). Then, the 2D drag coefficient is computed this way :\\
$ bfrcoef2d(:,:) = rn\_bfri2 * ( 1 + rn\_bfrien * bfrcoef2d(:,:) ) $ \\
In this simulation the amplification factor (rn\_bfrien) is set to 50, which imply the use of an implicit scheme for stability reasons. 

%-----------------------------------nambfr--------------------------------------------
\namdisplay{nambfr}
%-------------------------------------------------------------------------------------

\begin{figure}[H]
\begin{center}
\includegraphics[width=15cm]{./TexFiles/Figures/<CONFIG_CASE>_bfr_coef_bering}
\caption{Bottom friction coefficient in the vicinity of Bering Strait. Colors are bathymetry.}
\label{bfrcoef2}
\end{center}
\end{figure}
%
\begin{figure}[H]
\begin{center}
\includegraphics[width=15cm]{./TexFiles/Figures/<CONFIG_CASE>_bfr_coef_torres}
\caption{Bottom friction coefficient in the vicinity of Torres Strait. Colors are bathymetry.}
\label{bfrcoef1}
\end{center}
\end{figure}

\subsubsection{Tracer damping strategy}

The tracer damping strategy is common to all our DRAKKAR runs (nn\_hdmp=-2). Although the idea is to keep tracer damping as reduced as possible, 
it is necessary to have it in order to address the following 3 identified problems : \\
(1) Fix tracer trends in almost closed seas where the forcing is not trusted,
(2) fix spurious modification of water mass properties downstream of overflow regions and 
(3) reduce the long term trend in ACC transport, linked to the decrease of AABW volume.  \\
The Gouretski and Koltermann (2004) annual climatology is used; this climatology, build with isopycnal interpolation previous the regridding on z-level is
prefered to the standard Levitus climatology, especially for the Southern Ocean, were spurious density inversion in Levitus are noticeable. \\
In the DRAKKAR version of \textit{tradmp.F90}, \textit{resto} subroutine has been modified so that the definitions of the restoring zones are independant of
the model configuration and are given in geographical coordinates either as a rectangular shape or as a circular shape with localized center and given radius,
associated with a depth range.
This helps in maintaining a coherent strategy for all the DRAKKAR configurations. \\
\\
In the following paragraphs some details on the implementation of the tracer damping for the three points mentioned above:\\

\paragraph{Regional 3D damping (semi-enclosed seas) \\}

Table  \ref{table3Ddmp} gives the characteristics of the 3D TS damping zones.

\begin{table}[h]
\begin{center}
\begin{tabular}{|c|c|c|c|c|c|}
\hline
\textbf{Region} & \textbf{Longitude Range} & \textbf{Latitude Range}  & \textbf{Depth Range(m)}  & \textbf{ time scale (d)} \\
\hline
\textbf{Red Sea}         & 29.4 E - 43.6 E & 12.9 N - 30.3 N          &  0. - bottom             & 180.     \\
\textbf{Black Sea}       & 27.4 E - 42.0 E & 41.0 N - 47.5 N          &  0. - bottom             & 180.     \\
\textbf{Persian Gulf}    & 46.5 E - 57.0 E & 23.0 N - 31.5 N          &  0. - bottom             & 180.      \\
\hline
\end{tabular}
\label{table3Ddmp}
\caption{ Definition of 3D TS restoring zone.}
\end{center}
\end{table}

\paragraph{Downstream the overflows \\}
One of the major flaws of the numerical model is its incapacity of well representing the overflow of dense waters over sills, despites the BBL parameterization.
As a consequence of this flaw, the entrainement downstream the sill is far too much, leading to produce a too light (hence too shallow)  overflow water. \\

This effect is particularly dramatic for the mediterranean outflow at Gibraltar Strait: without damping, in general the Med Sea Water (MW) spreads out at 450-500m 
whereas it should get deep to 1200 m. After years of simulations, this has a huge impact on the water masses of the North Atlantic, which is 
not acceptable. In ORCA12.L46-MAL101 simulation, we found that at this resolution and with a precise local bathymetry, that preserves a natural submarine canyon,
we were able to significantly improve the quality (both in depth and density) of the MW overflow, but with the climatological forcing, (this run) it appeared 
that for a long integration it was not enough and we did not take the risk of "poluting" all the North Atlantic. Other similar regions such as Bab-el-Mandeb 
(conexion between the Red Sea and the Gulf of Aden) and Ormuz Strait (conexion between the Persian Gulf and the Arabian Sea) should behave in an almost similar
way and the same restoring strategy is used there. \\

The overflow of the Nordic Seas at Denmark strait and Faroes Bank channel, cannot be fixed so drastically, because on long term run, the interannual variability
(which is one of our scientific interest) may be driven by the variability of the overflow, which is killed by the TS restoring process.
\textcolor{blue}{ JMM : commentaires plus pertinent ??}\\

In the climatological run, we do observed the formation of a spurious polynia in the center of the Wedell Sea and as a consequence, a very deep convection (almost to the bottom) there, that drastically changed the properties of the deep waters. Many tests were performed (see section \ref{prod}) aiming at eliminating the polynia, but none worked satisfactory, except having a light TS restoring in the upper ocean. Due to the production context of the GENCI "grand challenge", with very
short available production time, we opt for keeping this light restoring.

 Table \ref{tableoverflowdmp} gives the characteristics of the overflow damping zone.
\begin{table}[h]
\begin{center}
\begin{tabular}{|c|c|c|c|c|c|}
\hline
\textbf{Region} & \textbf{Patch Center}        & \textbf{Patch Radius (km)}  & \textbf{Depth Range(m)}  & \textbf{ time scale (d)} \\
\hline
\textbf{Gibraltar Strait}  & 7.0 W - 36.0 N    &  80                       &  600 - 1300              & 6.     \\
\textbf{Bab-el-Mandeb}     & 44.75 E - 11.5 N  &  100                      &  0 - bottom              & 6.     \\
\textbf{Ormuz Strait}      & 57.75 E - 25.0 N  &  100                      &  0 - bottom              & 6.     \\
\hline
\hline
\textbf{  }        & \textbf{Longitude Range} & \textbf{Latitude Range}  & \textbf{Depth Range(m)}  & \textbf{ time scale (d)} \\
\hline
\textbf{Wedell Sea}    & 66 W - 15.0 E            & 75.0 S - 60.0 S          &  0. - 500             & 360.      \\
\hline
\end{tabular}
\label{tableoverflowdmp}
\caption{ Definition of overflow restoring zones.}
\end{center}
\end{table}


\paragraph{Antarctic Bottom Water restoring \\}

3D TS restoring (time scale of 2 years) in an area limited by the sigma-2=34.7 isopycnal, a depth greater than 1000m and south of 30S. \\

We are using ORCA12.L46\_dmp\_mask.nc as dmpmask.nc: i.e. \textit{includefile.ksh} file indicates:
\begin{verbatim}
AABW_DMP=1
WDMP=ORCA12.L46_dmp_mask.nc
\end{verbatim}

This ORCA12.L46\_dmp\_mask.nc file was created with \textit{cdfmaskdmp} cdftool using T/S Gouretski climatology 2004. \\

%-----------------------------------namtra_dmp--------------------------------------------
\namdisplay{namtra_dmp}
\namdisplay{namtsd}
%-------------------------------------------------------------------------------------

%---------------------------------------------------------------------------------------------------
