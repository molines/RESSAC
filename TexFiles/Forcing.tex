\section{Surface boundary conditions}

The surface boundary conditions are prescribed to the model using the CORE bulk formulation. 
The run was forced by DFS5.1.1 over the 1979-2010 period. \textcolor{blue}{donner une ref de raphael, donner une breve explication de qu'est ce qui differe entre ERAinterim et DFS5.1.1}. 
The data set includes 4 turbulent variables (u10, v10, t2, q2) given every 3 hours, 2 radiative fluxes variables (radsw, radlw) and 
2 fresh water flux variables (total precipitations, snow), all these last variables given as daily averaged. 
The turbulent variables (u10, v10, t2, q2) and the fresh water flux variables (total precipitations, snow) are time interpolated, 
contrary to the radiative fluxes variables (radsw, radlw) \textcolor{blue}{vous etes bien d'accord avec ces interpolations (je me souviens avoir foir\'e pour MAL95 sur ce point...) ?}. \\

\textcolor{blue}{Remark: Markus force K003 avec CORE2 et lui n'interpole pas t2 q2 u10 v10 en temps, contrairement a nous..., lui il a t10 q10 et pas t2 q2} \\

\textcolor{blue}{Modification de derni\`ere minute: Bernard veut encore changer le coefficient devant les pr\'ecipitations. Il serait temps de enfin finaliser ce forcage et d\'ecider qu'est ce qu'on y met !} \\


We run the simulation twice over the same forcing period 1979-2010: for a total of 32 $\times$ 2 = 64 simulated years. \\

The frequency of surface boundary condition computation (and also the frequency of sea-ice model call) is every 6 timesteps. 
This value was defined considering that:
\begin{itemize}
 \item the timestep duration is 360sec (after stabilization of the model);
 \item turbulent forcing are given every 3h;
 \item thus we call sea-ice model and surface boundary condition computation every 36min, i.e. 5 times between each forcing, which is sufficient.
\end{itemize}

\textcolor{blue}{Remark: Markus has nn\_fsbc=5 with rn\_rdt=540 in its K003 run.} \\

We used ice model LIM2-EVP. \\

We use absolute 10m wind (we don't take into account the surface current in the wind velocity module), except for the ice. \textcolor{blue}{Some results 
(reported here: \\
\url{http://www-meom.hmg.inpg.fr/DRAKKAR/WORK_AL/EKE_ABSREL_WIND/}) show that absolute wind associated with 50-level vertical grid results in a 
stronger EKE at the surface and in the first 100meters, 
compared to relative wind associated with 46-level vertical grid. Actually we don't know if it is absolute wind or high resolution near surface 
(or both) which is responsible for that but we decide to use absolute wind for ORCA12.L46-MAL101 simulation.}

%-----------------------------------namsbc--------------------------------------------
\namdisplay{namsbc}
\namdisplay{namsbc_core}
%-------------------------------------------------------------------------------------

\subsection{Radiative flux and precipitation corrections \\}

%The radiative fluxes (both long wave and short wave) and precipitation exhibit unacceptable bias. 
%Therefore a specific correction has been implemented by \textcolor{blue}{Garric (reference paper ?)} 
%in order to improve those fluxes and precipitation. For radiative fluxes, a correction factor based on 
%the comparison between ERA-interim fluxes and satellite fluxes products (GEWEX) is computed. For precipitation, 
%the correction uses large scale GPCP product. These corrections required some change in the code as described here. 
%Basically, the idea is to apply a 2D scaling coefficient to the large scale features of the radiative fluxes and precipitation. 
%Original fields are band-pass filtered to separate large scales and small scales, using a shapiro filter, applied 250 times during 
%the period 1989-1991 and enhanced to 600 times during the period 1992-2007 (which reduced the performance rate of the model computing by 5\%). 
%The correction is applied to the large scale and then the small scale is added to produce the radiative fluxes and precipitation for the model. 
%Finally we applied a scale factor to this radiative flux correction in Arctic and Southern Ocean (solar heat flux: 70\% in the Arctic and 80\% 
%in the Southern Ocean; long wave heat flux: 110\% in the Arctic and Southern Ocean). For precipitation, we only applied the correction between 
%30\degres S and 30\degres N, with a buffer zone between 30\degres N/S and 40\degres N/S. No correction is applied northward of 40\degres N 
%and southward of 40\degres S.\\

\textcolor{blue}{blahblahblah... Raf ???}

\subsection{Light penetration algorithm according to ocean color \\}

In this simulation we use a non-standard parametrization of the penetration of the solar flux in the ocean, 
modulated by the chlorophyll concentration, deduced from satellite (SeaWIFS ocean color products) ocean color 
monthly climatology developed by \cite{Lengaigne2007}. In this solar radiation penetration formulation, visible 
light is split into three wavebands: blue (400-500nm), green (500-600nm), red (600-700nm).\\

We take care to use nn\_chldta=1 (not the same error as in ORCA025.L75-GRD100, ORCA12.L46-MAL95, ORCA025.L75-MJM95 runs...). \\

\textcolor{blue}{Markus a gard\'e nn\_chldta=0 pour K003...} \\

The Chl-a data file on an ORCA12 grid was provided by Sebastien Masson: chlaseawifs\_c1m-99-05\_smooth\_ORCA\_R12.nc. 
We used kRGB61.txt 61-level RGB bands in the first 10 meters.

%-----------------------------------namtra_qsr--------------------------------------------
\namdisplay{namtra_qsr}
%-------------------------------------------------------------------------------------

\subsection{Diurnal Cycle on solar fluxes \\}

We used the parametrization of the diurnal cycle on the solar flux (ln\_dm2dc=.true.), although 
the ORCA12.L46-MAL101 simulation as only 46-level vertical discretization. We did this choice 
to be coherent with the corresponding twin ORCA025 simulation which has 75-level vertical grid. \\


\subsection{River Run-off \\}

The coastal and river run-off data is based on the file runoff\_obtaz\_rhone\_1m\_ORCA12\_20102008.nc
\footnote{It is the first ORCA12 simulation computed at LEGI with this runoff file, previous one used the 
runoff file with Rhone into the Garonne (runoff\_coast1pt\_ant3pt\_isl20\_obtaz\_1m\_ORCA12\_correctAMZ\_200610\_lbclnk.nc).}. \\

There is no specific treatment at rivers mouths. 
We don't read any information about depth / temperature / salinity for runoff. \\

The novelty is that we decide to take into account the melting of drifting icebergs. 
We start from the same runoff file used in ORCA025.L75-GRD100. We extract only the 
iceberg runoff (using BMG tool). 
We extent\footnote{no interpolation, one grid cell is copied into 9 grid cells} 
this ORCA025-grid iceberg runoff onto an ORCA12 grid (with chgrid.F90 tool). We compute 
the total iceberg runoff south of 60S and we remove this value to the Antarctic coastal 
runoff south of 60S, in order to keep the same total runoff as before. Table 
\ref{tablerunoff} shows the annual total runoff (global and south of 60S) in initial file, in only-iceberg file, and in final file. 
Figures \ref{runoffhistglo} and \ref{runoffhistS60} show the global and south of 60S cumulative runoff at each month in initial file, in only iceberg file, and in final runoff file. 
This final runoff file used in ORCA12.L46-MAL101 simulation is called runoff\_obtaz\_rhone\_1m\_ORCA12\_20102008\_iceberg.nc and is plotted in fig. \ref{runoff2D}. \\

\textcolor{blue}{Et pour la variable socoefr ?} \\

\begin{table}
\begin{center}
\begin{tabular}{|c|c|c|c|}
\hline
\textbf{Moy annuelle (Sv)} & \textbf{initial} & \textbf{iceberg} & \textbf{final} \\
\hline
\textbf{Global}            & 1.3147           & 0.0206           & 1.3201         \\
\textbf{South of 60S}      & 0.0828           & 0.0152           & 0.0828         \\
\hline
\end{tabular}
\label{tablerunoff}
\caption{Global and South of 60S runoff in initial runoff file (runoff\_obtaz\_rhone\_1m\_ORCA12\_20102008.nc), 
in only-iceberg runoff (deduced from ORCA025.L75-GRD100 runoff file: runoff\_GIG\_antarcoast\_corrected.nc), 
and in final runoff file used in ORCA12.L46-MAL101 simulation 
(runoff\_obtaz\_rhone\_1m\_ORCA12\_20102008\_iceberg.nc).}
\end{center}
\end{table}

\begin{figure}[H]
\begin{center}
\includegraphics[width=15cm]{FIGURES/globalrunoff.eps}
\caption{Global runoff (in Sv) for each month.}
\label{runoffhistglo}
\end{center}
\end{figure}

\begin{figure}[H]
\begin{center}
\includegraphics[width=15cm]{FIGURES/S60runoff.eps}
\caption{South of 60S runoff (in Sv) for each month. We can see that the south of 60S runoff is conserved after adding iceberg runoff.}
\label{runoffhistS60}
\end{center}
\end{figure}

\begin{figure}[H]
\begin{center}
\includegraphics[width=15cm]{FIGURES/runoff_obtaz_rhone_1m_ORCA12_20102008_iceberg_annual.eps}
\caption{Annual mean runoff (Kg/m2/s) in ORCA12.L46-MAL101 simulation.}
\label{runoff2D}
\end{center}
\end{figure}

%-----------------------------------namsbc_rnf--------------------------------------------
\namdisplay{namsbc_rnf}
%-------------------------------------------------------------------------------------

\textcolor{blue}{Once we considered patching runoff from MED12 into \\
runoff\_obtaz\_rhone\_1m\_ORCA12\_20102008.nc. But strong differences 
between the two files over the Mediterranean Sea (reported here: \\
\url{http://www-meom.hmg.inpg.fr/DRAKKAR/WORK_AL/RUNOFF_MED/}) prevented us to do that. 
Contrairement � nous, J. Beuvier place ses runoff \`a l'embouchure des fleuves, et pas \'etal\'e dans l'oc\'ean. De plus il n'a fait aucun tuning particulier pour le m\'elange \`a Gibraltar. 
Donc nous, qu'est ce qu'on d\'ecide ?}.\\

\textcolor{blue}{Question: ce run off n'est pas � jour du Amery Ice Shelf qu'on a bouch\'e, croyez vous que c'est grave ?... en meme temps 
je crois qu'il coincide pas avec le mask \`a plein d'autres endroits... je vais essayer d'analyser quels sont les endroits d'incoh\'erence et combien de Sv de runoff ca concerne...} \\

\subsection{SSS restoring strategy \\}

This run uses Sea Surface Salinity restoring toward Levitus, with a time scale of 60 days / 10 meters (considered as rather strong). 
There is no enhancement of the restoring term in the Mediterranean Sea. \\

\textcolor{blue}{Markus a rn\_deds=-16.44 dans K003, soit 608 days/10m (rappel 10 fois plus faible que nous)} \\

\textcolor{blue}{Decision: we use SSS damping under sea-ice, with the same time scale as for open ocean ? Question: comment fait on avec la namelist pour dire qu'on veut rappeler aussi sous la glace ?} \\

\textcolor{blue}{Question: should we bound erp term and which value for rn\_sssr\_bnd ?} \\

\textcolor{blue}{Question: Which mask near the coast ? I don't have any mask for ORCA12... We want restore near the coast 
northward of 55-65N, and no restoring near the coast elsewhere... And which rn\_dist ? En fait non, on ne sait pas du tout ce qu'on veut faire pour l'instant. Qui tranche ?} \\

\textcolor{blue}{Question: We decided to keep model SSS filtering before restoring (ln\_sssr\_flt=.true.), 
but we want an isotropic smoothing. (how to do that with current Shapiro filter ?). Thierry: ``SSS filtering : OK shapiro a condition que (1) on sache quel est son cutoff 
et que (2) il soit isotrope. Il me semble pertinent d'exprimer cette \'echelle de cutoff sur la grille i,j qui suit un peu la d\'ecroissance poleward du rayon interne (quoiqu'insuffisamment)''. 
OK, alors on fait quoi avec ce Shapiro ?} \\

\textcolor{blue}{Question: Bernard souhaite un rappel en sel tr\`es fort les 4 premi\`eres ann\'ees, puis on reviendrait (progressivement ou brusquement ?) vers 60 days / 10 meters. 
C'est \`a dire, quel time scale choisir exactement pour year 1 \`a 4 ?}

%-----------------------------------namsbc_ssr--------------------------------------------
\namdisplay{namsbc_ssr}
%-------------------------------------------------------------------------------------

